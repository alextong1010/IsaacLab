\documentclass{article}
\usepackage{listings}
\usepackage{xcolor}
\usepackage{hyperref}
\usepackage{geometry}

\geometry{margin=1in}

\definecolor{codegreen}{rgb}{0,0.6,0}
\definecolor{codegray}{rgb}{0.5,0.5,0.5}
\definecolor{codepurple}{rgb}{0.58,0,0.82}

\lstdefinestyle{mystyle}{
    commentstyle=\color{codegreen},
    keywordstyle=\color{blue},
    stringstyle=\color{codepurple},
    basicstyle=\ttfamily\small,
    breakatwhitespace=false,
    breaklines=true,
    keepspaces=true,
    showspaces=false,
    showstringspaces=false,
    numbers=left,
    numberstyle=\tiny\color{codegray},
    stepnumber=1,
    tabsize=4
}

\lstset{style=mystyle}

\title{Detailed Explanation of Velocity Environment Configuration}
\author{Isaac Lab Documentation}
\date{\today}

\begin{document}

\maketitle

\section{Introduction}
This document provides a detailed explanation of the velocity environment configuration file used in the Isaac Lab project. The configuration defines a reinforcement learning environment for training robots to perform velocity-based locomotion tasks.

\section{Scene Configuration}
The scene configuration (\texttt{MySceneCfg}) defines the physical environment where the robot operates:

\subsection{Terrain}
\begin{itemize}
    \item Uses a terrain generator with rough terrain settings
    \item Configurable friction and restitution properties
    \item Visual materials using marble textures
    \item Maximum initial terrain level set to 5
\end{itemize}

\subsection{Sensors}
The environment includes two main sensors:
\begin{itemize}
    \item \textbf{Height Scanner}: A ray-casting sensor that:
    \begin{itemize}
        \item Mounted on robot base
        \item Uses a grid pattern with 0.1m resolution
        \item Covers area of 1.6m x 1.0m
    \end{itemize}
    \item \textbf{Contact Forces}: Tracks:
    \begin{itemize}
        \item Contact forces on robot parts
        \item 3-frame history
        \item Air time tracking
    \end{itemize}
\end{itemize}

\section{MDP Components}

\subsection{Commands Configuration}
The \texttt{CommandsCfg} class defines how velocity commands are generated:
\begin{itemize}
    \item Base velocity commands with ranges:
    \begin{itemize}
        \item Linear velocity X: [-1.0, 1.0] m/s
        \item Linear velocity Y: [-1.0, 1.0] m/s
        \item Angular velocity Z: [-1.0, 1.0] rad/s
        \item Heading: [-π, π] rad
    \end{itemize}
    \item Resampling time: 10 seconds
    \item Heading control stiffness: 0.5
\end{itemize}

\subsection{Actions Configuration}
Defines how the robot can be controlled:
\begin{itemize}
    \item Joint position control
    \item Scaling factor of 0.5
    \item Uses default offset positions
\end{itemize}

\subsection{Observations Configuration}
The observation space includes:
\begin{itemize}
    \item Base linear and angular velocities
    \item Projected gravity vector
    \item Current velocity commands
    \item Joint positions and velocities
    \item Previous actions
    \item Height scan data
\end{itemize}

Each observation includes carefully tuned noise parameters for robustness.

\subsection{Events Configuration}
Events are divided into three categories:

\subsubsection{Startup Events}
\begin{itemize}
    \item Physics material randomization
    \item Base mass modifications
\end{itemize}

\subsubsection{Reset Events}
\begin{itemize}
    \item External force/torque application
    \item Base state reset
    \item Joint state reset
\end{itemize}

\subsubsection{Interval Events}
\begin{itemize}
    \item Periodic robot pushing (every 10-15 seconds)
\end{itemize}

\subsection{Rewards Configuration}
The reward function combines multiple terms:

\subsubsection{Task Rewards}
\begin{itemize}
    \item Tracking linear velocity (weight: 1.0)
    \item Tracking angular velocity (weight: 0.5)
\end{itemize}

\subsubsection{Penalty Terms}
\begin{itemize}
    \item Vertical linear velocity (weight: -2.0)
    \item XY angular velocity (weight: -0.05)
    \item Joint torques (weight: -1.0e-5)
    \item Joint accelerations (weight: -2.5e-7)
    \item Action rate (weight: -0.01)
    \item Feet air time (weight: 0.125)
    \item Undesired contacts (weight: -1.0)
\end{itemize}

\section{Environment Settings}
Key environment parameters:
\begin{itemize}
    \item 4096 parallel environments
    \item 2.5m spacing between environments
    \item 20-second episode length
    \item 5ms physics timestep
    \item 4x action decimation
\end{itemize}

\section{Curriculum}
The environment supports curriculum learning through:
\begin{itemize}
    \item Progressive terrain difficulty
    \item Automatic terrain level adjustment based on robot performance
\end{itemize}

\end{document} 